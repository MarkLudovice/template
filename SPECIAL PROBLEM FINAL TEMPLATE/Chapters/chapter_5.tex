\clearpage
\thispagestyle{empty}

\addcontentsline{toc}{chapter}{\hspace{6mm} \textbf {CHAPTER 5} \hspace{7mm} \textbf {SUMMARY, FINDINGS, CONCLUSIONS AND \\ \hspace*{3.9cm}  RECOMMENDATIONS} \hspace{6.3cm}}

\begin{center}
	\textbf{{CHAPTER 5}}\\
	\vspace{-1ex}
	\textbf{SUMMARY, FINDINGS, CONCLUSIONS AND RECOMMENDATIONS}
	\vspace{-2ex}
\end{center}

This chapter presents the Summary of Findings, Conclusions, and Recommendations of the study Topic Modelling on Mayon Volcano Tweets using Latent Dirichlet Allocation (LDA).

\subsection{Summary of Findings and Accomplishments}
\vspace{-2ex}Based on the objectives of the study, the following results were accomplished:

\vspace{-2ex}
\begin{enumerate}
	\item {\textbf{\textcolor{red}{The Develop Tool/Application/System.} Sample 1.} Integrating game elements in the developed 2 dimensional Digital Netiquette Gamification for android was able to capture users’ interests and increase their awareness about sensitive issues such as digital netiquette.\\
	\textbf{Sample 2. }The developed topic modeling tool was able to collect, pre-process, generate topic models as well as visualize these models that provided the unique meaning of the collected dataset.\\
	\textbf{Sample 3.} Utilizing online open-source dataset using getold tweets was able to produce enough data to generate appropriate models. A total of 38,888 tweets consisting of specified keywords and date were successfully collected.
	
	\item \textbf{\textcolor{red}{The algorithm used for the Developed System / Application to achieve the necessary functions or output/s of the study.} Sample 1.} Utilizing the Iterative Dichotomiser 3 (ID3) Algorithm through the incorporation of a decision tree and reward system was able to evaluate users’ learning progress.\\
	\textbf{Sample 2.} The Latent Dirichlet Allocation Algorithm was able to generate topic models by appropriately categorizing similar groups of data in a datasets to determine similar topics from the collected disaster-related responses.\\
	\textbf{Sample 3.} The Standard and Bidirectional Recurrent Neural Network was effective in terms of classifying typhoon related corpus and was able to generate relevant models.
	
	\item \textbf{\textcolor{red}{Assessing the users’ knowledge through pre and post evaluation test / Extent of correctness of the generated topic models based on its topic coherence / Performance Evaluation of the Classification Models using accuracy, precision F measures and recall metrics.}}\\
	\textbf{Sample 1.} The pre-evaluation and post-evaluation test was able to assess the users’ knowledge learned as applied in the real-situational scenarios before and after using the gamification.\\
	\textbf{	Sample 2.} Manual evaluation through human judgment was used to manually evaluate topic models by ranking topic significance as well as topic to words similarity. This provided a more inclusive result of the true quality of the different topic models.\\
	\textbf{Sample 3.} The performance of the classification models were determined by the standard metrics. It was observed that the two RNN algorithms produced closed evaluation results where Bidirectional RNN obtained accuracy rate of  81.67\%, 81.17 precision, 81.67\% recall and 80.81\% f-measure against performance evaluation result of regular RNN where is obtained 81.25\% accuracy, 80.84\% precision, 81.25\% recall and 80.25\% f-measure.
\end{enumerate}
\vspace{-2ex}

\subsection{ Conclusions}
\vspace{-2ex}Based from the findings, the researchers came up with the following conclusions:
\vspace{-2ex}
\begin{enumerate}
	\item \textbf{Sample 1} Integrating game elements such as trophies, badges, power ups, sounds and animation in the developed gamification increased the interests of users to use the application. \\
	\textbf{Sample 2.} The features of the proposed system can easily upload datasets to be cleaned, analyzed, processed in-order to generate appropriate topic models as well as visually present the topic models to better present and appreciate the generated results.\\
	\textbf{Sample 3.} Integrating the collection, pre-processing, generation of models, visualizing as well as evaluating the models in the developed tool provided an easier methods of Classifying datasets.
	
	\item \textbf{Sample 1.} Using a Decision Tree which utilizes the Iterative Dichotomiser 3 (ID3) Algorithm provided a real scenario based decision making application that measures users’ progress based on the provided storyline. \\
	\textbf{Sample 2.} The Latent Dirichlet Allocation (LDA) algorithm was able to generate topic models by appropriately categorizing similar groups of data in a datasets to determine similar topics from the collected disaster-related responses.\\
	\textbf{Sample 3.} The process of classification of Typhoon Yolanda related tweets with relatively high accuracy was made possible using the sequential process of standard and bidirectional Recurrent Neural Networks (RNN) and their powerful architecture which remembers the past and future context of data.
	
	\item \textbf{Sample 1.} The pre-evaluation and post-evaluation tests was able to determine whether the user gained knowledge in proper computer etiquette before and after using the application.\\
	\textbf{Sample 2.} The use of both standard and bidirectional recurrent neural network algorithms in sentiment analysis were effective since it was able to achieve scores higher than 80\% in all evaluation metrics.\\
	\textbf{Sample 3.} The result of the evaluation on the extent of correctness of the generated topic models based on the top-N words using model precision yielded 87\% mean denoting that the manual annotations of the generated topic models agreed with the human judgments.
	
\end{enumerate}

\noindent
\subsection{ Recommendations}
\vspace{-2ex}On the basis of the conclusions, the researchers come up with several recommendations.  
\vspace{-3ex}
\begin{enumerate}
	\item Other features, modules of the develop system / application / tool / datasets
	\item Include other possible algorithms, comparison of algorithms, other methods
	\item Other test metrics, higher / better models, better results
\end{enumerate}