\clearpage
\thispagestyle{empty}
\addcontentsline{toc}{chapter}{\hspace{6mm}\textbf{CHAPTER 1 \hspace{8mm} INTRODUCTION \hspace{7.7cm}}}

\begin{center}
	\textbf{{CHAPTER 1}}\\
	\vspace{-1ex}
	\textbf{INTRODUCTION} 
\end{center}
\subsection{Introduction}
\vspace{-3ex}
A few decades ago, most of the works were done manually. Because of the willingness to make those works easier, man has tried many ways to make those things happen. This is when technology starts to become more advanced as time goes by. Things like automation of work are becoming more in demand to companies and agencies, may it be private or public. Information technology has helped many people in terms of storing, retrieving, transmitting information, and in communicating. Today, people are still looking for ways and inventing things that will benefit those in the future generation and for the advancement in the field of information technology.

One of the advancement of technology is the Android--a Linux-based operating system designed primarily for touchscreen mobile devices such as smartphones and tablet computers.  This open source code and permissive licensing allows the software to be freely modified and distributed by device manufactures, wireless carries and enthusiast developers.  Such factor has contributed towards making Android--the world’s most widely used smartphone platform—and the software of choice for technology companies who require low-cost, customizable, lightweight operating system for high tech devices without developing one from scratch.

Game development is a creative method or process that combines computer programming with animation, graphics, sounds within a certain period of time to develop an interactive game through the computer. These interactive games can be a means of entertaining ourselves and is popular for both children and adults.

Computer simulation has become a useful part of modeling many natural systems in physics, chemistry and biology, and human systems in economics and social science (the computational sociology) as well as in engineering to gain insight into the operation of those systems. A good example of the usefulness of using computers to simulate can be found in the field of network traffic simulation. In such simulations, the model behaviour will change each simulation according to the set of initial parameters assumed for the environment.

%Purpose and Description of the Project (Do not show)

Rubik's Cube is a 3-D combination puzzle invented in 1974 by Hungarian sculptor and professor of architecture Ernő Rubik.  It is widely considered to be the world's best-selling toy.
In a classic Rubik's Cube, each of the six faces is covered by nine stickers, each of one of six solid colours (traditionally white, red, blue, orange, green, and yellow, where white is opposite yellow, blue is opposite green, and orange is opposite red, and the red, white and blue are arranged in that order in a clockwise arrangement).  An internal pivot mechanism enables each face to turn independently, thus mixing up the colours.  For the puzzle to be solved, each face must be returned to consisting of one colour.  Similar puzzles have now been produced with various numbers of sides, dimensions, and stickers, not all of them by Rubik.

The proposed study will focus on the usage of an artificial intelligence that can solve the 3x3x3 Rubik’s Cube.  The Android device may visualize the arrangement of colors of the cube via user input.  After the device imaged the cube, it will then try to solve it by implementing a solution through a simulation of the Rubik’s Cube that will help users ease the difficulty of solving the puzzle game.


\subsection{Significance of the Research}
\vspace{-3ex}
Playing computer games are good recreational activity but not all could enhance player’s strategic thinking. Deflexion game enhances player’s strategic thinking while having fun. This study is directed to create a 3D deflexion game that could be played in a LAN which is beneficial to:

\textbf{Rubik’s Cube Enthusiasts.} This application will be a way of connecting those people who love the cube to the new technology and engage them in a way that is exciting and new.

\textbf{Mobile Application Aficionado.} Definitely one of the apps that the Android people will surely want to have on their collection.

\textbf{Android Developer.} This will serve as a basis for other Android Developers to learn something from.

\textbf{Researchers.} The researchers can learn immensely on the development of this application.  This may be used as a stepping stone to aim for the job that the researchers certainly want.  

\textbf{Future Researcher.} This will be of great motivation to the future researchers / neophyte inventors for them to pursue their ideas no matter how intimidating it may sound.

%------------------------Objectives of the Project----------------
\subsection{Objectives of the Project}
\vspace{-3ex}
The main objective of this research is to develop an Android application that can generate solution for a regular 3x3x3 Rubik's Cube. Specifically it attempted to answer the following objectives:\\
\vspace{-5ex}	
	\begin{enumerate} 
		\item To construct a GUI that simulates a blank cube and let the user fill up the face of each cubelet manually which will be capable of:
		\begin{enumerate}
			\item [a.]providing step by step instruction to solve the cube;
			\item [b.]provide game elements such as animations, sounds, and rotation option for the cube;
			\item [c.]allow game saving;
		\end{enumerate}
		\item To create an artificial intelligence that will apply the Two-Phase Algorithm to find one of the most efficient ways to solve the cube; 
		\item To measure and assess the Two-Phase Algorithm in the implementation of the 
		Robik’s Cube solver.	
	\end{enumerate}

%------------------------Scope and Limitations of the Project----------------
\subsection{Scope and Limitations of the Project}
\vspace{-3ex}
On our input page, the GUI will simulate a "blank cube" where-in the user will fill up the face of the cubelets to represent the real cube that they have. To fill up the face of the cubelets, the user will have to select the appropriate color in the color palette and apply it to specific cubelets. Each colors on the palette must only be used eight times.

One thing to note in each face of the cube is that the center colors--that color determines the face of the cube--are already fixed in place and immovable. That is to say that the face which has the white center and the color on it will be the White Face. For better orientation, the White Face’s top center color must be the color Blue, Red Face is White, Orange Face is White, Yellow Face is Blue, Green Face is White, and Blue Face is White. We will also have a random button to randomize the content of the cube.

The researcher will include animation and rotation of the cube in the solution window so that the users will find it easy to follow the steps to be taken to solve the cube. The proponents will create an Artificial Intelligence using Kociemba’s Two-Phase Algorithm that is specifically created for solving the 3x3x3 Cube. The algorithm solves the Cube in to steps. In phase 1, the algorithm looks for maneuvers which will transform a scrambled cube to G1. That is, the orientations of corners and edges have to be constrained and the edges of the Up and Downslice have to be transferred into that slice. In phase 2 we restore the cube. There are many different possibilities for maneuvers in phase 1. The algorithm tries different phase 1 maneuvers to find a most possible short overall solution.

%-----------------Definition of Terms--------------

\subsection{Definition of Terms}
\vspace{-3ex}
The following terms related to the research are defined operationally for better understanding:


\noindent\textbf {Android - } A Linux-based operating system designed primarily for touchscreen mobile devices such as smartphones and tablet computers.

\noindent\textbf{Rubik’s Cube - } A 3-D combination puzzle invented in 1974 by Hungarian sculptor and professor of architecture Ernő Rubik. Originally called the "Magic Cube".

\noindent\textbf{Puzzle - } is a problem or enigma that tests the ingenuity of the solver.  In a basic puzzle, one is intended to put together pieces in a logical way in order to come up with the desired solution.  Puzzles are often contrived as a form of entertainment, but they can also stem from serious mathematical or logistical problems — in such cases, their successful resolution can be a significant contribution to mathematical research.

\noindent\textbf{Application - } also known as application software or an app, is computer software designed to help the user to perform specific tasks.

